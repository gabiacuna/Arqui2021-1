\documentclass[12pt]{article}

\usepackage[utf8]{inputenc}
\usepackage{graphicx}
\usepackage{geometry}
\usepackage{titlesec}
\usepackage{xcolor}
\usepackage{listings}
\usepackage{changepage}
\usepackage{hyperref}

\usepackage{multicol}
\setlength{\columnsep}{1cm}
\linespread{1.0}
\renewcommand{\contentsname}{Índice}
\usepackage{mathptmx}


\definecolor{light-gray}{gray}{0.95}
\lstset{language=Python,keywordstyle={\bfseries \colorbox{light-gray}}}


\begin{document}

\begin{titlepage}
    \centering

    \begin{figure}[t!]
        \centering
      \includegraphics[keepaspectratio, width=0.6\textwidth]{logoUSM.png}
    \end{figure}
    %\vspace{2cm}
    {\scshape\Large Informe 2, Arquitectura y Organización de Computadores. \par}
    \vspace{1.5cm}
    {\scshape\Huge Implementación logica de bool y sistemas combinacionales para construir circuitos.  \par}
    \vspace{1.5cm}
    %{\itshape\Large Sobre sistemas numéricos,  y representación de datos.\par}
    %\vfill
    {\Large Gabriela Acuña, rol 201973504-7.\par}
    \vspace{0.2cm}
    {\Large Profesor Mauricio Solar.}
    \vfill
    {\Large  6 de Junio, 2021 \par}

\end{titlepage}

\tableofcontents
\clearpage


\section{Resumen}
%donde describa brevemente el desarrollo y resultados de la tarea.


%\begin{multicols}{2}
\section{Introducción}
%dejando claro el objetivo de la tarea y cualquier fórmula que utilice


 
\section{Desarrollo}
%donde describa brevemente el desarrollo y resultados de la tarea.


\section{Resultados}


\section{Análisis}
%Los resultados encontrados son los esperados para todas las bases y códigos, considerando que cuando un código se encontraba en la posición de b, se tomaba n como un numero codificado en el mismo.


%\end{multicols}

\section{Conclusión}

%comentando el nivel de finalización de la tarea. Además explique generalmente la función de las distintas bases numéricas y códigos de corrección estudiados en clases.


\newpage

\section{Anexo}

\end{document}